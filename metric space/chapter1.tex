\documentclass[12pt,letterpaper]{article}
\usepackage{fontspec}
\usepackage{amsmath}
\usepackage{enumerate}
\usepackage{multicol}

\setmainfont{Times New Roman}
\title{Metric Space}
\author{Yang}
\date{\today}

\begin{document}
\maketitle
\tableofcontents


\section{Metric Space}

\subsection{Definition}
A metric space is a pair (\textit{X},\textit{d}), where \textit{X} is a set and d is a 
metric on \textit{X} (or \textit{distance function} on \textit{X}), that is, a
function defined on \textit{X} $\times$ \textit{X} such that for all x,y,z $\in$ \textit{X} we have:
\begin{itemize}
\item[1.] \textit{d} is real-valued, finite and nonnegative. 
\item[2.] \textit{d}(x,y) = 0 \textbf{if and only if } x = y. 
\item[3.] \textit{d}(x,y) = \textit{d}(y,x)
\item[4.] \textit{d}(x,y) $\leq$ \textit{d}(x,z) + \textit{d}(z,y)
\end{itemize}

\subsection{Subspace}
A \textbf{subspace} (\textit{Y},$\widetilde{\textit{d}}$) of (\textit{X},d) is obtained if
we take a subset \textit{Y} $\subset$ \textit{X} and restrict \textit{d} to
$Y \times Y$ is the restriction
$$ \widetilde{d} = d \lvert_{Y \times Y} $$
$\widetilde{\textit{d}}$ is called the metric induced on \textit{Y} by \textit{d}.

\subsection{Examples}
\begin{tabular}{cc}
Real line R & \textit{d}(x,y) = $\lvert x - y \rvert$ \\
Euclidean space $R^n$, unitary space $C^n$, complex C & \textit{d}(x,y) = $\sqrt{
    (\xi_1 - \eta_1)^2 + \dots + (\xi_n-\eta_n)^2
    }$\\
Sequence space $l^{\infty}$(bounded) & $\textit{d}(x,y)=\sup\limits_{j \in N} \lvert
\xi_j - \eta_j \rvert$\\
Function space C[a,b].(continuous) & $\textit{d}(x,y)=\max\limits_{t \in J}\lvert
x(t)-y(t)\rvert$ \\
Discrete metric space & $d(x,x)=0$,  $d(x,y)=1$  $(x\neq y)$\\
Sequence space s & $
d(x,y) = \sum\limits_{j=1}^{\infty} \frac{1}{2^j} \frac{\lvert \xi_j - \eta_j
\rvert}{1+\lvert \xi_j - \eta_j \rvert}
$\\
Space B(A) of bounded functions & $d(x,y)=\sup\limits_{t\in A}
\lvert x(t)-y(t)\rvert$ \\
Space $l^p$ & $d(x,y)=(\sum\limits_{j=1}^{\infty}{\lvert \xi_j - \eta_j \rvert}
^p)^{1/p}$ 
\end{tabular}

\section{Open set, Closed set, Neighborhood}
\subsection{Ball and Sphere}
Definition: Given a point $x_0 \in X$ and a real number $r > 0$, we define
three types of set:
\begin{enumerate}
\item[(a)] $B(x_0;r) = \{x\in X \vert d(x,x_0) < r\}$  (Open shell) 
\item[(b)] $\widetilde{B}(x_0;r) = \{x\in X \vert d(x,x_0) \leq r\}$  (Closed shell)
\item[(c)] $S(x_0;r) = \{x\in X \vert d(x,x_0) = r\}$  (Sphere)
\end{enumerate}
We see that an open ball of radius $r$ is the set of all points in $X$ whose
distance from the center of the ball is less than $r$. Furthermore, the definition immediately
implies that
$$ S(x_0;r)=\widetilde{B}(x_0;r)-B(x_0;r)$$

\subsection{Open set and Closed set}
A subset $M$ of a metric space $X$ is said to be \textit{\textbf{open}} if it contains a ball
about each of its points.\\
A subset $K$ of $X$ is said to \textit{\textbf{closed}} if its complement (in $X$) is
open, that is, $K^C =X-K$ is open.\\
An open ball $B(x_0;\epsilon)$ of radius $\epsilon$ is often called an 
\textit{\textbf{$\epsilon$ - neighborhood}} of $x_0$.\\

\end{document}
